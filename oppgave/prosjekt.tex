\documentclass[11pt,a4paper]{article}

\usepackage[utf8]{inputenc}
\usepackage[T1]{fontenc}
\usepackage[norsk]{babel}
\usepackage{amssymb}
\usepackage{amsmath}
\usepackage{listings}
\usepackage{enumitem}
\usepackage{color}
\usepackage{hyperref}

\frenchspacing

\definecolor{grey}{rgb}{.95, .95, .95}
\lstset{basicstyle=\footnotesize\ttfamily,language=Python,backgroundcolor=\color{grey}}

\begin{document}

\title{«Poisson Image Editing»}
\author{Individuell prosjektoppgave \\
  i IMT3881 Vitenskapelig programmering}
\date{Våren 2018}

\maketitle

\section{Metoden i generelle trekk}

En rekke problemer i bildebehandling kan løses med en teknikk som
kalles «Poisson Image Editing»~\cite{Perez:03}. Metoden går i korthet
ut på at man representerer bildet man ønsker å komme frem til som en
funksjon $u : \Omega \to C$, der $\Omega \subset \mathbb{R}^2$ er det
rektangulære området hvor bildet er definert, og $C$ er fargerommet,
vanligvis $C = [0, 1]$ for gråtonebilder og $C = [0, 1]^3$ for
fargebilder. Bildet fremkommer som en løsning av Poisson-ligningen
$$
\frac{\partial^2 u}{\partial x^2} + \frac{\partial^2 u}{\partial y^2}
\equiv \nabla^2 u = h,
$$
der randverdier på $\partial\Omega$ og funksjonen
$h : \Omega \to \mathbb{R}^{\dim(C)}$ spesifiseres avhengig av hvilket
problem som skal løses.

En måte å løse Poisson-ligningen på er å iterere seg frem til
løsningen vha. såkalt gradientnedstigning («gradient descent»). I
praksis gjøres dette ved å innføre en kunstig tidsparameter og la
løsningen utvikle seg mot konvergens:
\begin{equation}
\frac{\partial u}{\partial t} = \nabla^2 u - h.
\label{eq:diffusjon}
\end{equation}
Når man velger denne fremgangsmåten, må man også velge en initialverdi
for bildet, $u(x, y, 0) = u_0(x, y)$.

To diskrete numeriske skjemaer for~(\ref{eq:diffusjon}) kan finnes ved
henholdsvis eksplisitt og implisitt tidsintegrasjon og sentrerte
differanser for de spatielle deriverte:
\begin{align}
  \frac{u^{n+1}_{i,j} - u^n_{i,j}}{\Delta t} &= \frac{1}{\Delta x^2}
                                               (u^n_{i+1,j} +
                                               u^n_{i-1,j} +
                                               u^n_{i,j+1} +
                                               u^n_{i,j-1} -4 
                                               u^n_{i,j}) - h_{i,j},
                                               \label{eq:eksplisitt}  \\
  \frac{u^{n+1}_{i,j} - u^n_{i,j}}{\Delta t} &= \frac{1}{\Delta x^2}
                                               (u^{n+1}_{i+1,j} +
                                               u^{n+1}_{i-1,j} +
                                               u^{n+1}_{i,j+1} +
                                               u^{n+1}_{i,j-1} -4 
                                               u^{n+1}_{i,j}) - h_{i,j}.
                                               \label{eq:implisitt}
\end{align}
Det eksplisitte skjemaet~(\ref{eq:eksplisitt}) er noenlunde rett frem
å implementere, mens det implisitte~(\ref{eq:implisitt}) er noe mer
krevende. Sistnevnte løses enklest ved å skrive det på matriseform og
bruke rutiner for glisne matriser i implementasjonen.

\section{Anvendelser}
\label{sec:anvendelser}

Her følger en kort beskrivelse av noen av anvendelsene av «Poisson
Image Editing».

\subsection{Glatting}
\label{sec:glatting}

Hvis vi tar utgangspunkt i et originalbilde $u_0(x, y)$, kan man
implementere glatting («blurring») av bildet ved å iterere
ligning~(\ref{eq:diffusjon}) med $h = 0$ i hele $\Omega$. Bildet vil
da bli stadig glattere (mer uskarpt) med tiden $t$. Som randverdier
kan man enten bruke Diriclet med $u(x, y, t) = u_0(x, y)$ på
$\partial\Omega$ (gir litt skarpere kant enn strengt tatt nødvendig),
eller, bedre, Neumann med $\partial u/\partial n = 0$ på
$\partial\Omega$ (symmetri).

\subsection{Inpainting}
\label{sec:inpainting}

Hvis vi enten mangler noe informasjon i et bilde $u_0(x, y)$, eller
ønsker å fjerne noe fra det (støy, tekst som er lagt oppå el.l.), kan
vi gjøre dette ved å fylle inn («inpaint») informasjon i gjeldende
område basert på informasjonen rundt området. Hvis
$\Omega_i \subset \Omega$ er området som skal fylles inn, kan dette
gjøres ved å sette $h = 0$ i $\Omega_i$ og løse
ligning~(\ref{eq:diffusjon}) i $\Omega_i$ med Dirichlet-betingelsen
$u(x, y, t) = u_0(x, y)$ på $\partial\Omega_i$.

For å implementere dette er det hensiktsmessig å innføre en maske i
form av en boolsk array som er sann for alle pixler som er innenfor
$\Omega_i$ og usann for alle som er utenfor. En slik array kan da
benyttes som index for et
«view»\footnote{\url{https://docs.scipy.org/doc/numpy/user/basics.indexing.html}}
av bildet, og gjøre operasjoner bare på pixlene innenfor eller
utenfor. F.eks. kan alle pixlene utenfor $\Omega_i$ settes lik
verdiene fra originalbildet ved å skrive \texttt{u[\textasciitilde
  omega\_i] = u\_0[\textasciitilde omega\_i]}, der \texttt{omega\_i}
er den boolske arrayen (masken).

Dersom masken berører kanten av bildet, trenger vi en randbetingelse
på $\partial\Omega$ også. Dette kan da gjøres på samme måte som for
glatting (se over).

\subsection{Kontrastforsterkning}
\label{sec:kontrastforsterkning}

Jo større den lokale kontrasten i et bilde $u$ er, desto større er
gradienten til bildet, $\nabla u$. For å finne en mer kontrastert
utgave av originalbildet $u_0$, kan vi altså forsøke finne et bilde
som har samme gradient som $u_0$, men forsterket med en konstant
$k > 1$. Dette kan gjøres ved å sette $h = k\nabla^2 u_0$ inn
i~(\ref{eq:diffusjon}) og løse for $u$. Hensiktsmessige randverdier er
enten Dirichlet, $u(x, y, t) = u_0(x, y)$ eller, bedre, Neumann,
$\partial u/\partial n = k\partial u_0/\partial n$ på
$\partial\Omega$. Merk at iterering av~(\ref{eq:diffusjon}) med
$k > 1$ fort kan føre til løsninger med $u > 1$ eller $u < 0$, altså
utenfor det tilgjengelige fargeområdet. Det må derfor innføres som en
føring at $u \in [0, 1]$. Dette kan i praksis gjøres i koden ved å
klippe verdiene for $u$ til intervallet i slutten av hver iterasjon.

En mer avansert form for kontrastforsterkning kan vi lage som
beskrevet i~\cite{Fattal:02} ved å innføre en ikkelineær funksjon av
gradienten, f.eks. $g = f(\nabla u_0)$, og så la
$h = \nabla\cdot g = \nabla\cdot(f(\nabla u_0))$ i
ligning~(\ref{eq:diffusjon}). Randverdier og føringer blir som
beskrevet over.

\subsection{Demosaicing}
\label{sec:utvidelse_1}

Bildesensoren i et digitalkamera er egentlig monokrom, og kan bare
måle mengden lys som faller inn på den i hver pixel. For å kunne lage
fargebilder, legger man en mosaikk av fargefiltere over den, slik at i
hver pixel måles i praksis kun én av fargekanalene, f.eks. R, G og B.
Og ut av sensoren kommer det altså en gråtonemosaikk.

En slik gråtonemosaikk kan man simulere i Python ved å ta utgangspunkt
i et fargebilde \texttt{u} representert ved en $M\times N \times 3$
\texttt{numpy array}. Gråtonemosaikken kan lages som følger:
\begin{lstlisting}
mosaic = np.zeros(u.shape[:2])         # Alloker plass
mosaic[ ::2,  ::2] = u[ ::2,  ::2, 0]  # R-kanal
mosaic[1::2,  ::2] = u[1::2,  ::2, 1]  # G-kanal
mosaic[ ::2, 1::2] = u[ ::2, 1::2, 1]  # G-kanal
mosaic[1::2, 1::2] = u[1::2, 1::2, 2]  # B-kanal
\end{lstlisting}

Oppgaven til en demosaicing-algoritme er å rekonstruere et fargebilde
ut av en slik gråtonemosaikk. Én måte å gjøre dette på, er å først
flytte informasjonen som finnes i mosaikken over i de rette kanalene i
et fargebilde, for deretter å «inpaint-e» den manglende informasjonen
vha. inpaintingsmetoden beskrevet over. Det må da altså lages en
$\Omega_i$ for hver kanal som definerer pixlene som skal fylles inn.

\subsection{Sømløs kloning}

Noen ganger ønsker man av ymse grunner å kunne flytte en del av et
bilde inn i et annet bilde på en slik måte at det ikke blir synlige
overganger der objektet er limt inn. Dette kalles gjerne sømløs
kloning. Hvis vi kaller originalbildet det skal klones inn i for
$u_0$, og bildet som det skal klones inn fra for $u_1$, kan dette
formuleres som et Poisson-problem: Finn $u$ slik at $u = u_0$ for
$x \not\in \Omega_i$ og $\nabla^2 u = \nabla^2 u_1$ i $\Omega_i$.
Dette kan altså gjøres ved å sette $h = \nabla^2 u_1$ og
løse~(\ref{eq:diffusjon}) i $\Omega_i$ med Diriclet-betingelsen
$u = u_0$ på $\partial\Omega_i$. Også her må man huske å klippe slik
at $u \in [0, 1]$.

For implementasjonen i Pyton bør man også merke seg at $\Omega_i$ ikke
nødvendigvis behøver å befinne seg på samme sted i $u_0$ og $u_1$.
Dette kan løses ved bruk av «views» på \texttt{numpy array}-ene.

\subsection{Konvertering av fargebilder til gråtone}

Den vanligste måten å konvertere fargebilder til gråtone er ved å ta
et (veiet) gjennomsnitt av R, G og B-kanalene. Når fargebilder
konverteres til gråtonebilder på denne måten, kan det lett skje at noe
av informasjonen i bildet forsvinner. Særlig gjelder dette der det er
detaljinformasjon (teksturer, kanter etc.) mellom elementer som i
hovedsak skiller seg i fargetone og/eller metning, men som i hovedsak
har samme lyshet.

En litt mer sofistikert teknikk går ut på å søke å konstruere et
gråtonebilde som har så lik lokal variasjon som det originale
fargebildet som mulig. Dette kan gjøres ved å konstruere en ny
gradient $g$ gitt med lengde $||\nabla u_0||/\sqrt{3}$ (hvorfor
$\sqrt{3}$?) og retning som $\nabla(u_{0R} + u_{0G} + u_{0B})$, og så la
$h = \nabla\cdot g$ og løse ligning~(\ref{eq:diffusjon}) for $u$ i
hele $\Omega$. Her vil det være flere muligheter for spesifikasjon av
randbetingelsene. En naturlig initialverdi vil være gjennomsnittet av
de tre fargekanalene i originalbildet.

En enda mer sofisikert teknikk for definisjon av $h$ er ved bruk den
såkalte strukturtensoren til fargebildet, se~\cite{Alsam:08}.

\subsection{Rekonstruksjon og visualisering av HDR-bilder}

Scener i dagslys har ofte en ekstremt høy dynamikk, det vil si
forskjell mellom mørkeste og lyseste punkt i scenen som skal avbildes.
Ofte er denne dynamikken så høy at selv en god bildesensor ikke klarer
å registrere hele dynamikken, og vi får bilder som samtidig er både
over- og undereksponert, altså bilder der noen områder er helt svarte,
og andre er helt hvite. En teknikk for å unngå dette for statiske
scener, er å ta flere bilder (med kameraet på stativ) med ulik
eksponeringstid, og så sette dem sammen til et HDR-bilde («High
Dynamic Range») i ettertid. Dette gjøres ved at man estimerer
kamera-respons-kurven samtidig som man estimerer den faktiske lysheten
i scenen som et stort, sammensatt minste-kvadraters problem,
se~\cite{Debevec:97}.

Når man så har fått et slik HDR-bilde støter man på et nytt problem,
nemlig at de fleste enheter for visning av bilder (skjermer,
projektorer, skrivere etc.) heller ikke har stor nok dynamikk til å
vise dem frem. Så dynamikken i bildet må altså komprimeres igjen før
det kan vises og lagres i vanlige 8-bits-formater. Dette kan gjøres på
akkurat motsatt måte av kontrastforsterkningen vi innførte i andre del
av avsnitt~\ref{sec:kontrastforsterkning} med et passende valg av
gradient-komprimeringsfunksjon, se~\cite{Fattal:02}

\subsection{Anonymisering av bilder med ansikter}

Noen ganger trenger man å anonymisere personene i et bilde før det
vises frem offentlig. En måte å gjøre dette på, er å gjøre ansiktene
uskarpe mens resten av bildet beholdes skarpt. Hvis man har en maske
$\Omega_i$ som beskriver områdene i bildene som inneholder ansikter,
kan dette gjøres ved å løse ligning~(\ref{eq:diffusjon}) med $h=0$ i
$\Omega_i$ og Dirichlet-betingelsen $u = u_0$ på $\partial\Omega_i$.
Utfordringen er å finne masken. Her kan kanskje
OpenCV\footnote{\url{https://opencv.org/}} være til hjelp.

\subsection{Kantbevarende glatting}
\label{sec:utvidelse_end}

Noen ganger ønsker man å glatte et bilde uten å gjøre kantene i bildet
uskarpe. Dette kan gjøres ved å innføre en posisjonsavhengig
diffusjonsparameter $D : \Omega \to [0, 1]$, der D = 0 gir null
diffusjon/glatting og $D = 1$ gir full glatting. Diffusjonsparameteren
kan f.eks. beregnes fra gradienten til bildet som
\begin{equation}
  D(x, y) = \frac{1}{1 + \kappa ||\nabla u_0(x, y)||^2}\,,
\end{equation}
med $\kappa$ som en passende valgt konstant.
Diffusjonsligningen~(\ref{eq:diffusjon}) endres da til
\begin{equation}
  \label{eq:lokal}
  \frac{\partial u}{\partial t} = \nabla\cdot(D\nabla u),
\end{equation}
og løses for $\Omega$ med f.eks. Dirichletbetingelsen
$u(x, y) = u_0(x, y)$ på $\partial\Omega$. Merk at det da må lages
egne numeriske skjemaer for denne modifiserte ligningen. Merk også at
hvis $D=1$ overalt, vil~(\ref{eq:lokal}) bli identisk
med~(\ref{eq:diffusjon}) med $h=0$.

Denne utvidede modellen kan også på forskjellig vis brukes til å
forbedre oppførselen av flere av de tidligere anvendelsene, så her
åpnes et hav av nye muligheter\dots

\section{Oppgave}

\subsection{Gjennomføring}

«Fork» bitbucket-repoet på vanlig måte: privat fork, men med arvede
egenskaper, slik at kun du selv og emneansvarlig har innsyn. La både
kode og rapport bo i repoet. Gjør hyppige og små nok commits til
repoet slik at det blir mulig å følge utviklingen av prosjektet i
ettertid, også når man ikke har en fungerende løsning (bruk gjerne
«branches»). Det vil altså i praksis si flere commits enn man strengt
tatt ville gjort i en realistisk utviklingssituasjon.

\subsection{Minimumsløsning}

Implementer det eksplisitte skjemaet~(\ref{eq:eksplisitt}) og
anvendelsene glatting (avsnitt~\ref{sec:glatting}), inpainting
(avsnitt~\ref{sec:inpainting}) og kontrastforsterkning for
gråtonebilder (avsnitt~\ref{sec:kontrastforsterkning}) beskrevet over.
Beskriv problemstillingen, løsningen og resultater i form av
eksempelbilder i en velformet rapport. Rapporten skal inneholde en
lenke til det «forkede» repoet.

\subsection{Utvidelser}

Det er ubegrenset med muligheter for å utvide besvarelsen og dermed
gjøre den bedre:
\begin{itemize}[noitemsep]
\item Implementer løsningen også for fargebilder (enkel liten
  utvidelse).
\item Implementer de øvrige anvendelsene beskrevet i
  avsnitt~\ref{sec:utvidelse_1}--\ref{sec:utvidelse_end} (de er
  stort sett beskrevet i rekkefølge av økende kompleksitet).
\item Implementer det implisitte numeriske
  skjemaet~(\ref{eq:implisitt}) vha. glisne matriser («sparse
  matrices»)\footnote{\url{https://docs.scipy.org/doc/scipy/reference/sparse.html}}
  og prøv det for alle de implementerte anvendelsene.
\item Lag en applikasjon med grafisk brukergrensesnitt som gir
  brukeren mulighet til å utføre alle de implementerte operasjonene
  interaktivt. Bruk f.eks.
  PyQt5.\footnote{\url{https://www.riverbankcomputing.com/software/pyqt/intro}}
\end{itemize}

\subsection{Vurderingskriterier}

Ved vurderingen av prosjektoppgaven vil det bli lagt vekt på
\begin{itemize}[noitemsep]
\item Hvilke metoder som er implementert
  \begin{itemize}[noitemsep]
  \item Gråtonebilder/fargebilder
  \item Eksplisitt/implisitt etc.
  \item Typer randverdier for $\partial\Omega$ og $\partial\Omega_i$
  \end{itemize}
\item Hvilke anvendelser som er implementert
  \begin{itemize}[noitemsep]
  \item Glatting
  \item Inpainting
  \item Kontastforsterkning (flere mulige varianter)
  \item Demosaicing
  \item Sømløs kloning
  \item Farge til gråtone-konvertering
  \item Rekonstruksjon og visualisering av HDR-bilder 
  \item Anonymisering av bilder med ansikter
  \item Kantbevarende glatting
  \item GUI-applikasjon
  \end{itemize}
\item Kvalitet på koden, herunder
  \begin{itemize}[noitemsep]
  \item Struktur og gjenbruk av kode
  \item Gjenbrukbarhet i form av moduler
  \item Dokumentasjon i form av velformede doc-strings
  \item Variabelnavn
  \item Automatiserte (enhets)tester
  \end{itemize}
\item Kvalitet på rapporten, herunder
  \begin{itemize}[noitemsep]
  \item Struktur
  \item Språk
  \item Formler
  \item Referanser
  \item Kryssreferanser
  \item Figurer
  \item Tabeller
  \item Kodelistinger
  \end{itemize}
\item Prosessen (slik den fremkommer av git-historikken)
\end{itemize}

\section{Innlevering}

Rapporten innleverers som PDF i Inspera innen fredag 27. april 2018
kl. 1600. Rapporen \emph{skal} inneholde en lenke til Bitbucket-repoet.
Bitbucket-repoet må minimum få leve til etter at sensuren er gitt (og
lenger dersom man skal kunne klage på karakteren).

\bibliographystyle{unsrt}
\bibliography{prosjekt}

\end{document}
